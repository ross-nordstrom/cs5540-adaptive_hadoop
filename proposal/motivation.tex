\section{Motivation}
\label{section:motivation}
Hadoop MapReduce is an easy-to-use implementation of distributed, parallel data 
processing. It has a history of making it easy for data scientists and industry 
developers to implement parallel computing in the cloud, but for the serious 
user of Hadoop, there is room for improvement. Hadoop does most things well, 
but we propose a modification to it that would improve its performance for all 
users.

\subsection{Homogeneous Task Assignment}
Hadoop MapReduce assumes homogeneous hardware configurations for the nodes in 
its cluster when it schedules and assigns tasks to them. If the nodes in the 
cluster in fact have heterogeneous hardware configurations then the execution 
and completion times of these tasks could vary in unanticipated ways. 
Additionally, interference from non-cluster nodes running on the same hardware 
could change performance as well. These two problems can cause inefficient 
resource usage and degraded performance. 

\subsection{Modifying Hadoop}
A new scheduling and assignment algorithm is needed that is capable of 
identifying the hardware configurations of the nodes in its cluster and 
scheduling tasks on them accordingly. This algorithm should also be sensitive 
to interference from other non-cluster nodes and adapt. The hardware 
configurations of the cluster nodes can be set ahead of time, however, 
interference from non-cluster nodes cannot be anticipated. This algorithm 
should be able to detect interference and dynamically change the task settings 
accordingly.

The existing Hadoop task scheduler/assignment module will need to be identified 
in the source code for Hadoop 1.2.1. It will then need to be either modified or 
replaced to meet the requirements proposed in this document. A server cluster 
with heterogeneous hardware configurations will be created and used to test the 
performance of the system. Both the modified and unmodified code will be built 
and deployed on this cluster. Both will run the same MapReduce job and the 
performance of each will be compared.
