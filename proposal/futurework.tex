\section{Future Work}
\label{section:futurework}

There is some remaining work to be done to fully qualify our Adaptive Hadoop idea. Additionally, 
assuming our idea holds, there are a number of potential extensions of our work. In section we
describe both sets of future work.

\subsection{Immediate Work}
To fully vet out our proposal, we still need to assign the modified input splits to a node
of relative size and evaluate the performance difference to standard Hadoop with this change.
Once this is complete, we would setup a test suite to automate performance evaluations as we
develop an algorithm to qualify a node's performance capabilities and map that to a modified
input split size. We would then need to customize the \texttt{JobConf} class to carry
information about nodes (hosts) necessary to determine split sizes, assign these classifications
to the\texttt{TaskTrackerStatus} for each node and schedule map tasks with input sizes
appropriately. We envision this information could be as simple as a key-value list tying
a performance multiplier to each node (1x, 2x, 3x), relative to the least performant node.

\subsection{Extension Work}
Another approach to balancing the execution of MapTasks would be to calculate input splits
dynamically instead of creating them all ahead of time. When a slot is open on a \texttt{TaskTracker}, it would
be assigned a task with an input split that gets created with a size appropriate for the current circumstances
and the processing capabilities of the node it will be executed on.

As mentioned earlier, interference from third party VMs running on the same hardware
as the Hadoop nodes can cause their performance to vary and create execution imbalances
between nodes. This is true for both
homogeneous and heterogeneous clusters. To make Hadoop truly adaptive, it should
have some way to account for this interference when creating splits or scheduling tasks. A possible
solution to this was alluded to in previous sections: if input sizes could be split dynamically, 
and/or nodes could provide current performance metric information to the JobTracker in the
\texttt{heartbeat} then the \texttt{TaskScheduler} could create correctly sized splits and 
assign tasks accordingly.
