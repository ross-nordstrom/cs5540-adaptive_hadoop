\documentclass{sig-alternate}

\usepackage{graphicx}


\usepackage{url}
\usepackage{hyperref} 
\hypersetup{breaklinks} 

\begin{document}
\title{Adaptive and Heterogeneous Hadoop MapReduce}
\author{Aaron Burkhart, Ross Nordstrom\\
        University of Colorado - Colorado Springs\\
        1420 Austin Bluffs Pkwy,\\
        Colorado Springs, CO 80918\\
        \texttt{\{aburkhar,rnordstr\}@uccs.edu}
       }
\date{April 2014}

\maketitle

\begin{abstract}
Hadoop is a commonly used Apache implementation of MapReduce, a technique for
parallel processing in distributed systems. One weakness of Hadoop is it does
not assign tasks based on the computers’ computational capabilities. By scaling
how much work is sent to a given computer based on its capabilities, we can
redce the difference in task completion time across the system, and thus improve
the overall performance of Hadoop. We call such a system – with computers of
varying capabilities – a heterogeneous environment. This research aims to develop
a task scheduling and assignment algorithm for Hadoop MapReduce that is sensitive 
to both the hardware configurations of the nodes in the cluster and interference
from other non-cluster nodes running on the same hardware as cluster node.
\end{abstract}

\category{C.2.4}{Performance}{Cloud computing}
\terms{Performance, Design}
\keywords{Hadoop, MapReduce, Heterogeneous, Configuration, Cloud}

\section{Introduction}
Hadoop MapReduce is well-liked for its ability to dispatch tasks across workers
in a datacenter. Hypothetically, if the work is split up into equal parts and dispatched
efficiently, the system will work well. In reality, datacenters today are being
updated continuously with new hardware, while old hardware lingers in the datacente
. Additionally, most consumers of datacenters are not the owners. Take Amazon EC2
or Windows Azure, for example. Companies maintain datacenters and sell access to
it as a service. 

Because modern datacenters are so frequently consumed as a service, the consumer
has little control over what hardware to which they will have access. Even if the
consumer had full control over the which hardware they accessed, it is costly and
inefficient to maintain a datacenter full of machines with identical hardware. 
Additionally, creating an algorithm to deploy a given MapReduce job across only 
identical machines would be difficult.

Rather than fight the structure of datacenters, adapting Hadoop to the environment
in which it is deployed would allow for optimal MapReduce performance in arbitrary
environments. This is especially important when deploying Hadoop using an IaaS 
(infrastructure as a service).

Our research aims to modify Hadoop v1.2.1 to adaptively dispatch MapReduce tasks
based on both the workers’ computational power and scheduling status.

Example reference\ref{section:motivation}.

\section{Motivation}
\label{section:motivation}
Hadoop MapReduce is an easy-to-use implementation of distributed, parallel data 
processing. It has a history of making it easy for data scientists and industry 
developers to implement parallel computing in the cloud, but for the serious 
user of Hadoop, there is room for improvement. Hadoop does most things well, 
but we propose a modification to it that would improve its performance for its
users in a heterogeneous environment.

\subsection{Homogeneous Task Assignment}
Hadoop MapReduce assumes homogeneous hardware configurations for the nodes in 
its cluster when it schedules and assigns tasks to them. If the nodes in the 
cluster in fact have heterogeneous hardware configurations then the execution 
and completion times of these tasks could vary in unanticipated ways. 
Additionally, interference from non-cluster nodes running on the same hardware 
could change performance as well. These two problems can cause inefficient 
resource usage and degraded performance. 

\subsection{Modifying Hadoop}
A new input splitting and task assignment algorithm is needed that is capable of 
identifying the hardware configurations of the nodes in its cluster, computing
input splits of different sizes, and scheduling tasks on them accordingly.
An additional optimization not covered in this paper would be to modify 
this algorithm to also be sensitive 
to interference from other non-cluster nodes and adapt. The hardware 
configurations of the cluster nodes can be set ahead of time or queried at startup, however, 
interference from non-cluster nodes cannot be anticipated.

The existing Hadoop task scheduler/assignment module will need to be identified 
in the source code for Hadoop 1.2.1. It will then need to be either modified or 
replaced to meet the requirements proposed in this document. A server cluster 
with heterogeneous hardware configurations will be created and used to test the 
performance of the system. Both the modified and unmodified code will be built 
and deployed on this cluster. Both will run the same MapReduce job and the 
performance of each will be compared.

\section{Related Work}
\label{section:relatedwork}
Hadoop MapReduce is a popular subject of research. The primary focus of the research
is on performance improvement. Contributions other authors have made vary in
approach and effectiveness.

Tarazu is one framework that takes a similar approach to optimizing MapReduce \cite{Tarazu}.
It achieves a performance improvement over standard Hadoop by focusing on reducing network
interference during the Map phase and reducing disparity in the time for each node to complete
the Reduce phase. Tarazu differentiates between high performing nodes and low power nodes of
heterogeneous clusters. For the Map phase, it attempts to reduce the number of tasks running
remotely for high performing nodes and alleviates bursty network traffic. It reduces competition
for network bandwidth between these remote tasks and Shuffle tasks. Additionally, in the Reduce phase it attempts to
assign keys to Reduce tasks based on their hardware capabilities.

While similar to our work, Tarazu fails to assign work in anticipation of the nodes' computational
capability. This is a critical shortcoming in their design because it sets the system up
for needing to adjust tasks, and requiring the more performant nodes to steal more tasks
and do more remote work. Tarazu's authors cite the reason for this is to help improve data
reliability because they are preventing the large nodes from being a single point of failure.
We contend that this concern is not worth the performance they sacrifice by not loading the
large nodes appropriately. This concern could also be alleviated by providing some data duplication
for the input splits.

Here, we list a few of the other approaches to performance improvement that we have found:
\begin{description}
  \item{Levitated Merge:} Improves the performance of MapReduce by focusing on network latency between workers \cite{LevitatedMerge}.
  \item{SHadoop:} Improves MapReduce performance by optimizing the job and task execution \cite{SHadoop}.
  \item{HJ-Hadoop:} Improves MapReduce performance by optimizing for multi-core machines \cite{HJHadoop}.
  \item{I/O-Intensive Hadoop:} Improves MapReduce performance by using dynamic processing slots for I/O intensive jobs \cite{IOIntensiveHadoop}.
  \item{Hybrid MapReduce:} Improves MapReduce performance by using a combination of virtual and native environments \cite{HybridMR}.
\end{description}

We believe our Adaptive Hadoop solution will provide significant performance improvements
that could be used alongside many of these related works for additional gains. Additionally,
we have not found any work that takes an approach like ours to solve the heterogeneous
environment problem.
\section{Proposed Design}
\label{section:propeseddesign}

Before doing any implementation, we conducted some investigation into how Hadoop works, 
specifically how it assigns tasks initially and how they are reassigned during runtime.
These are important attributes to understand, because they contribute to uneven effective
workloads in a heterogeneous environment; that is, if two machines are assigned equal-sized
tasks, but one machine is twice as powerful as the other, it was effectively given half as
much work since it will complete it twice as fast as the other machine.

\subsection{MapReduce Task Assignment}
In standard Hadoop MapReduce, the Mapper divides up and distributes tasks to the workers.

\section{Evaluation}
\label{section:evaluation}
...[placeholder]...

\section{Future Work}
\label{section:futurework}

There is some remaining work to be done to fully qualify our Adaptive Hadoop idea. Additionally, 
assuming our idea holds, there are a number of potential extensions of our work. In section we
describe both sets of future work.

\subsection{Immediate Work}

\subsection{Extension Work}
Another approach to balancing the execution of MapTasks would be to calculate input splits
dynamically instead of creating them all ahead of time. When a slot is open on a \texttt{TaskTracker}, it would
be assigned a task with an input split that gets created with a size appropriate for the current circumstances
and the processing capabilities of the node it will be executed on.

\section{Conclusion}
\label{section:conclusion}
Our project was a difficult one, simply because building and deploying
Hadoop MapReduce is not trivial. Setting up our environment took up a
significant amount of our time. A benefit of this difficulty is the
extra work we did up front helped us gain a better understanding of
Hadoop.

The work we achieved was, we believe, a solid step towards proving our
hypothesis that modifying input split sizes based on the capabilities
of nodes in the Hadoop environment can optimize performance in heterogeneous
environments.

Our work left off on the cusp of vetting the idea. The major remaining
pieces to be done (described in \ref{section:futurework}) could be done without
significantly more work, and would be enough to validate the idea with
some benchmarking.

\bibliography{citations}{}
\bibliographystyle{plain}

\end{document}

