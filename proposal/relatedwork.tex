\section{Related Work}
\label{section:relatedwork}
Hadoop MapReduce is a popular subject of research. The primary focus of the research
is on performance improvement. Contributions other authors have made vary in
approach and effectiveness.

Tarazu is one framework that takes a similar approach to optimizing MapReduce \cite{Tarazu}.
It achieves a performance improvement over standard Hadoop by focusing on reducing network
interference during the Map phase and reducing disparity in the time for each node to complete
the Reduce phase. Tarazu differentiates between high performing nodes and low power nodes of
heterogeneous clusters. For the Map phase, it attempts to reduce the number of tasks running
remotely for high performing nodes and alleviates bursty network traffic. It reduces competition
for network bandwidth between these remote tasks and Shuffle tasks. Additionally, in the Reduce phase it attempts to
assign keys to Reduce tasks based on their hardware capabilities.

While similar to our work, Tarazu fails to assign work in anticipation of the nodes' computational
capability. This is a critical shortcoming in their design because it sets the system up
for needing to adjust tasks, and requiring the more performant nodes to steal more tasks
and do more remote work. Tarazu's authors cite the reason for this is to help improve data
reliability because they are preventing the large nodes from being a single point of failure.
We contend that this concern is not worth the performance they sacrifice by not loading the
large nodes appropriately. This concern could also be alleviated by providing some data duplication
for the input splits.

Here, we list a few of the other approaches to performance improvement that we have found:
\begin{description}
  \item{Levitated Merge:} Improves the performance of MapReduce by focusing on network latency between workers \cite{LevitatedMerge}.
  \item{SHadoop:} Improves MapReduce performance by optimizing the job and task execution \cite{SHadoop}.
  \item{HJ-Hadoop:} Improves MapReduce performance by optimizing for multi-core machines \cite{HJHadoop}.
  \item{I/O-Intensive Hadoop:} Improves MapReduce performance by using dynamic processing slots for I/O intensive jobs \cite{IOIntensiveHadoop}.
  \item{Hybrid MapReduce:} Improves MapReduce performance by using a combination of virtual and native environments \cite{HybridMR}.
\end{description}

We believe our Adaptive Hadoop solution will provide significant performance improvements
that could be used alongside many of these related works for additional gains. Additionally,
we have not found any work that takes an approach like ours to solve the heterogeneous
environment problem.